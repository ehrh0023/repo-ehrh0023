\hypertarget{index_intro_sec}{}\section{Introduction}\label{index_intro_sec}
As an invaluable means to practice the concepts we are learning in class, we are engaging in a semester long software development project in which we are building a robot simulator. We are using the Iterative Method. This means that software requirements are being given to us in stages. Each stage (iteration) builds on the previous one by increasing its functionality and possibly modifying some of the previous requirements.

Our assignment is to construct a program that meets the functional requirements and demonstrates good software development practices. \char`\"{}\-Good\char`\"{} means clear, concise, thoughtful, and readable code that is well documented and thoroughly tested. The quality of our software is equally, if not more important, than its functionality. In this situation, we are quite confident that by utilizing good development practices, good functionality will naturally follow.

We are writing the simulator using C++ and the Open\-G\-L, G\-L\-U\-T, and G\-L\-U\-I Utility graphics environments. (The graphics are as simplistic as they can be.) We have already started to use these libraries in the labs, and, in addition to the many on-\/line resources, we will continue to receive instruction in labs and in lectures. Open\-G\-L, G\-L\-U\-T, and G\-L\-U\-I operate on any and all platforms. We must ensure this software works on C\-S\-E machines.\hypertarget{index_Execution}{}\section{Execution}\label{index_Execution}
To run the robot simulator, execute the following commands\-:


\begin{DoxyCode}
make clean
make all
./gorobot
\end{DoxyCode}


To run the test cases, execute the following commands\-:


\begin{DoxyCode}
make clean
make all
./testrobot
\end{DoxyCode}
 